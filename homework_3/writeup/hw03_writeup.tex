
\documentclass[11pt]{exam} % https://www.ctan.org/pkg/exam?lang=en

\usepackage[lmargin=1.in,rmargin=1.in,tmargin=1.in,bmargin=1in]{geometry}
\usepackage{setspace}
\usepackage[pdftex]{graphicx}
\usepackage{titling}
\usepackage[
	pdfauthor={Brian Weinstein},
	pdftitle={Homework 3},
	bookmarks=true,
	colorlinks=true,
	linkcolor=blue,
	urlcolor=blue,
	citecolor=blue,
	pdftex,
	linktocpage=true
	]{hyperref}
\usepackage[textsize=tiny]{todonotes}
\usepackage{float}
\setlength\parindent{0pt}
\usepackage{lipsum}


\qformat{\textbf{Problem \thequestion: \thequestiontitle}\quad \hfill}


\pagestyle{headandfoot}
\runningheadrule
\firstpageheader{}{}{}
\runningheader{\thetitle}{\theauthor}{\thedate}
\firstpagefooter{}{\thepage}{}
\runningfooter{}{\thepage}{}


\usepackage{xcolor}
\usepackage{adjustbox}
\usepackage{verbatim}
\definecolor{shadecolor}{rgb}{.9, .9, .9}

\newenvironment{code}%
   {\par\noindent\adjustbox{margin=1ex,bgcolor=shadecolor,margin=0ex \medskipamount}\bgroup\minipage\linewidth\verbatim}%
   {\endverbatim\endminipage\egroup}

\newenvironment{codeSmall}%
   {\par\noindent\adjustbox{margin=1ex,bgcolor=shadecolor,margin=0ex \medskipamount}\bgroup\minipage\linewidth\verbatim\footnotesize}%
   {\endverbatim\endminipage\egroup}





\begin{document}


\title{STAT S4240 002, Homework 3}
\author{Brian Weinstein (bmw2148)}
\date{July 30, 2015}
\maketitle



\begin{questions}



\titledquestion{Naive Bayes Text Classification: Data Preparation}

See \texttt{hw03\_q1.R} for code.

\begin{parts}

\part Text pre-processing
\begin{codeSmall}
# load functions from hw03.R
source("hw03.R")

# preprocess text
preprocess.directory("datasets/FederalistPapers/fp_hamilton_train")
preprocess.directory("datasets/FederalistPapers/fp_hamilton_test")
preprocess.directory("datasets/FederalistPapers/fp_madison_train")
preprocess.directory("datasets/FederalistPapers/fp_madison_test")
\end{codeSmall}

\part Loading the cleaned data
\begin{codeSmall}
hamilton.train <- read.directory("datasets/FederalistPapers/fp_hamilton_train_clean")
hamilton.test <- read.directory("datasets/FederalistPapers/fp_hamilton_test_clean")
madison.train <- read.directory("datasets/FederalistPapers/fp_madison_train_clean")
madison.test <- read.directory("datasets/FederalistPapers/fp_madison_test_clean")
\end{codeSmall}


\part Create a dictionary from all of the documents
\begin{codeSmall}
dictionary <- make.sorted.dictionary.df(c(hamilton.train, hamilton.test,
                                          madison.train, madison.test))
\end{codeSmall}




\part Creating document-term-matrices for each of the datasets
\begin{codeSmall}
dtm.hamilton.train <- make.document.term.matrix(infiles=hamilton.train,
                                                dictionary=dictionary)
dtm.hamilton.test <- make.document.term.matrix(infiles=hamilton.test,
                                               dictionary=dictionary)
dtm.madison.train <- make.document.term.matrix(infiles=madison.train,
                                               dictionary=dictionary)
dtm.madison.test <- make.document.term.matrix(infiles=madison.test,
                                              dictionary=dictionary)
\end{codeSmall}







\part Compute the log probabilities for the dictionary in each of the document datasets
\begin{codeSmall}
mu=1/nrow(dictionary)

logp.hamilton.train <- make.log.pvec(dtm.hamilton.train, mu)
logp.hamilton.test <- make.log.pvec(dtm.hamilton.test, mu)
logp.madison.train <- make.log.pvec(dtm.madison.train, mu)
logp.madison.test <- make.log.pvec(dtm.madison.test, mu)
\end{codeSmall}

\end{parts}



\titledquestion{Naive Bayes Function}

We first estimate the log priors based on the log of the proportion of training documents attributed to each author.

$$p(\textrm{author}=\texttt{author})=\log\left(\frac{\textrm{\# of training documents attributed to }\texttt{author}}{\textrm{total \# of training documents}}\right)$$

Then, using (1) the log probabilities for the dictionary in a Hamilton-authored document and (2) the log probabilities for the dictionary in a Madison-authored document (as computed in \textbf{Problem 1}), we can input a new document-term-matrix and classify each document as belonging to one of the authors.

\begin{code}
naive.bayes <- function(logp.hamilton.train, logp.madison.train,
                        log.prior.hamilton, log.prior.madison, dtm.test){
  # Performs naive bayes classification
  # Inputs:  logp.hamilton.train  :   vector of log probabilities of words
  #                                     occurring in the hamilton training data
  #          logp.madison.train   :   vector of log probabilities of words
  #                                     occurring in the madison training data
  #          log.prior.hamilton   :   the log prior of hamilton documents
  #          log.prior.madison    :   the log prior of madison documents 
  #          dtm.test             :   a document-term-matrix to classify
  # Output:  Classification labels for each document in dtm.test
  
  # calculate the log posterior probabilities
  log.post.hamilton <- log.prior.hamilton + (dtm.test %*% logp.hamilton.train)
  log.post.madison <- log.prior.madison + (dtm.test %*% logp.madison.train)
  
  # compare the log posterior probabilities and assign to the author
  # with highest probability
  prediction <- data.frame(logPostHam=log.post.hamilton,
                           logPostMad=log.post.madison)
  prediction$pred <- (log.post.hamilton >= log.post.madison)
  prediction$pred <- gsub(TRUE, "Hamilton", prediction$pred)
  prediction$pred <- gsub(FALSE, "Madison", prediction$pred)
  
  # return a vector of the predictions
  return(prediction$pred)
  
}
\end{code}

\titledquestion{question 3}

Using the \texttt{confusionMatrix} function from the \texttt{caret} library:

\begin{itemize}
\item \textbf{Accuracy:} 63\% accurate (\% of the test papers that are classified correctly)
\item \textbf{True Positive Rate:} 100\% (Hamilton classified as Hamilton divided by the total amount of testing Hamilton papers)
\item \textbf{True Negative Rate:} 9\% (Madison classified as Madison divided by the total amount of testing Madison papers)
\item \textbf{False Positive Rate:} 91\% (Madison classified as Hamilton divided by the total amount of testing Madison)
\item \textbf{False Negative Rate:} 0\% (Hamilton classified as Madison divided by the total amount of testing Hamilton)
\end{itemize}


\begin{code}
> confusionMatrix(data=predictions$pred,
+                 reference=predictions$trueValue,
+                 dnn=c("Prediction", "True Value"),
+                 positive="Hamilton")
Confusion Matrix and Statistics

          True Value
Prediction Hamilton Madison
  Hamilton       16      10
  Madison         0       1
                                         
               Accuracy : 0.6296         
                 95% CI : (0.4237, 0.806)
    No Information Rate : 0.5926         
    P-Value [Acc > NIR] : 0.427258       
                                         
                  Kappa : 0.106          
 Mcnemar's Test P-Value : 0.004427       
                                         
            Sensitivity : 1.00000        
            Specificity : 0.09091        
         Pos Pred Value : 0.61538        
         Neg Pred Value : 1.00000        
             Prevalence : 0.59259        
         Detection Rate : 0.59259        
   Detection Prevalence : 0.96296        
      Balanced Accuracy : 0.54545        
                                         
       'Positive' Class : Hamilton       
\end{code}



\end{questions}

%\listoftodos

\end{document}